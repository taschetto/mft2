\documentclass[12pt]{article}

\usepackage{sbc-template}

\usepackage{graphicx,url}

\usepackage{fancyvrb}
\usepackage{listings}
\lstset {
    mathescape,
    frame=none
}

\usepackage{amsthm}
\usepackage{amssymb}
\usepackage{dsfont}

\usepackage[brazil]{babel}
\usepackage[utf8]{inputenc}
\usepackage{boxproof}

\sloppy

\title{Prova de Algorítmos}

\author{Pedro Vanzella\inst{1}, Guilherme Taschetto\inst{1}}

\address{Faculdade de Informática -- Pontifícia Universidade Católica do Rio
Grande do Sul (PUCRS) \\ Av. Ipiranga, 6681 - Porto Alegre / RS / Brasil
    \email{pedro@pedrovanzella.com, guilherme.taschetto@acad.pucrs.br}}

\begin{document}

\maketitle

\begin{abstract}
    This excercise aims to prove the validity of the invariants and Hoare
    triples for two algorithms, using induction and formal logical proofs.
\end{abstract}

\begin{resumo}
    Este exercício se propõe a provar a validade das invariantes e triplas de
    Hoare para dois algorítmos, usando indução e provas formais.
\end{resumo}

\section{Algoritmo 1}\label{sec:algo1}
Vamos definir uma invariante e prová-la, bem como as triplas de Hoare, para o
algoritmo a seguir:
\begin{lstlisting}
i := 1; j := 4;
while i $\neq$ x
    j := j + 2 * i + 3;
    i := i + 1
end
\end{lstlisting}

\subsection{Prova da Invariante}\label{sec:algo1:invar}
Devemos provar que
\[\forall n:\mathds{N}. j_n = (i_n + 1)^2 \wedge i_n \leq x\]

De acordo com o teorema de distributividade do para-todo, podemos separar
a invariante em duas partes e prová-las separadamente.

\subsubsection{}

\paragraph{Caso base} \[P(0) \triangleq j_0 = (i_0 + 1)^2 \]
Veja que:
\begin{proofbox}
  \:(i_0 + 1)^2 = (1 + 1)^2   \= i:=1   \\
  \:= 4                       \= arit   \\
  \:= j_0                     \= j:=4   \\
\end{proofbox}

\paragraph{Caso indutivo} \[P(k) \triangleq j_k = (i_k + 1)^2 \rightarrow
P(k+1) \triangleq j_{k+1} = (i_{k+1} +1)^2\]
Veja que:

\section{Conclusão}
\paragraph{}
Insira conclusão aqui.
É eeeeeasy. Tendeuquequedizê? Mecânicow.

\bibliographystyle{sbc}
\bibliography{teste}

\end{document}
