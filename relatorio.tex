\documentclass[12pt]{article}

\usepackage{sbc-template}

\usepackage{graphicx,url}

\usepackage{fancyvrb}
\usepackage{listings}
\lstset {
    mathescape,
    frame=none
}

\usepackage{amsthm}
\usepackage{amssymb}
\usepackage[fleqn]{amsmath}
\usepackage{dsfont}

\usepackage[brazil]{babel}
\usepackage[utf8]{inputenc}
\usepackage{boxproof}

\sloppy

\title{Prova de Programas Imperativos}

\author{Guilherme Taschetto\inst{1}, Pedro Vanzella\inst{1}}

\address{Faculdade de Informática -- Pontifícia Universidade Católica do Rio
Grande do Sul (PUCRS) \\ Av. Ipiranga, 6681 - Porto Alegre / RS / Brasil
    \email{guilherme.taschetto@acad.pucrs.br, pedro@pedrovanzella.com}}

\begin{document}

\maketitle

\begin{abstract}
    This article aims to prove the partial correctness for two imperative programs
    using Natural Induction and Hoare Logic as tools.
\end{abstract}

\begin{resumo}
    Este artigo se propõe a provar a correção parcial de dois programas imperativos
    usando como ferramentas para tal a Indução Natural e a Lógica de Hoare.
\end{resumo}

\section{Algoritmo 01}\label{sec:algo1}
\begin{lstlisting}
$Algoritmo\ 01\ (inteiro\ positivo\ x)$
$begin$
$variaveis\ locais$
$\ \ \ \ inteiros\ i,\ j;$
$\ \ \ \ i := 1; j := 4;$
$\ \ \ \ while\ i \neq x$
$\ \ \ \ \ \ \ \ j := j + 2 * i + 3;$
$\ \ \ \ \ \ \ \ i := i + 1$
$end$
\end{lstlisting}

Através da definição do cabeçalho do programa, podemos inferir a pré-condição da Tripla de Hoare correspondente: $(|\ x > 0\ |)$. Ainda, através de um teste de
mesa podemos levantar hipóteses a respeito do que o programa faz e a invariante do laço.

\begin{table}[h]
\begin{tabular}{|l|l|l|l|l|l|}
\hline
$n$ & $x$ & $i$ & $j $ & $i \neq x$ & $j=(i+1)^2 \land i \leq x $ \\ \hline
$0$ & $4$ & $1$ & $4 $ & $1 \neq 4$ & $4=(1+1)^2 \land 1 \leq 4 $ \\ \hline
$1$ & $4$ & $2$ & $9 $ & $2 \neq 4$ & $9=(2+1)^2 \land 2 \leq 4 $ \\ \hline
$2$ & $4$ & $3$ & $16$ & $3 \neq 4$ & $16=(3+1)^2 \land 3 \leq 4$ \\ \hline
$3$ & $4$ & $4$ & $25$ & $4 = 4   $ & $25=(4+1)^2 \land 4 \leq 4$ \\ \hline
$-$ & $4$ & $4$ & $25$ & $-       $ & $25=(4+1)^2 \land 4 \leq 4$ \\ \hline
\end{tabular}
\end{table}

Analisando a tabela gerada, fica evidente que o programa tem como objetivo realizar o cálculo de $(x+1)^2$ e armazenar o resultado na variável $j$. Disto podemos concluir que a pós-condição
da Tripla de Hoare é $(|\ j = (x+1)^2 \land i = x\ |)$. Além disto, encontramos um bom candidato para a invariante do laço: $j=(i+1)^2 \land i \leq x $. Para ter certeza que a
invariante é válida, é necessário realizar a prova formal da mesma.

\subsection{Tripla de Hoare}

\begin{lstlisting}
$(|\ x > 0\ |)$
$i := 1;$
$j := 4;$
$while\ i \neq x$
$\ \ \ \ j := j + 2 * i + 3;$
$\ \ \ \ i := i + 1$
$(|\ j = (x+1)^2 \land i = x\ |)$
\end{lstlisting}

\subsection{Prova da Invariante}\label{sec:algo1:invar}
Devemos provar que a invariante permanece válida durante a execução do laço, não importando o número de iterações. Ou seja, devemos provar que:
\[\forall n:\mathds{N}. j_n = (i_n + 1)^2 \wedge i_n \leq x\]

Aplicando o teorema da distributividade do operador $\forall$\footnote{$\forall x.P(x) \land Q(x) = \forall x.P(x) \land \forall x.Q(x)$},
podemos dividir a prova da invariante em duas partes: $\forall n:\mathds{N}. j_n = (i_n + 1)^2$ e $\forall n:\mathds{N}. i_n \leq x$. À partir disto, podemos aplicar técnicas de indução natural.

\subsubsection{Prova - Parte I}

Queremos provar que $\forall n:\mathds{N}. j_n = (i_n + 1)^2$. A proposição pode ser reescrita da forma:
\[P(k) \triangleq j_k = (i_k + 1)^2\]

\paragraph{Caso base} Precisamos provar que $P(0) \triangleq j_0 = (i_0 + 1)^2$
Veja que:
\begin{proofbox}
  \:(i_0 + 1)^2 = (1 + 1)^2   \= i:=1   \\
  \:= 4                       \= arit   \\
  \:= j_0                     \= j:=4   \\
\end{proofbox}
\hfill $\square$

\paragraph{Caso indutivo} Assumimos $j_k = (i_k + 1)^2$ como Hipótese de Indução ($HI$). Logo, precisamos provar que: \[P(k+1) \triangleq j_{k+1} = (i_{k+1} +1)^2\]
Agora, veja que:
\begin{proofbox}
  \:j_{k+1} = j_k + 2*i_k + 3 \= j:=j+2*i+3 \\
  \:= j_k + i_k + i_k + 3     \= arit   \\
  \:= j_k + i_k + i_k + 1 + 2 \= arit   \\
  \:= j_k + i_k + i_{k+1} + 2 \= i:=i+1 \\
  \:= j_k + i_{k+1} + i_{k+1} + 1 \= i:=i+1 \\
  \:= (i_k + 1)^2 + i_{k+1} + i_{k+1} + 1 \= HI \\
  \:= i_{k+1}^2 + 2i_{k+1} + 1 \= i:=i+1 \\
  \:= (i_{k+1} + 1)^2 \= arit \\
\end{proofbox}
\hfill $\square$

\subsubsection{Prova - Parte II}

Queremos provar que $\forall n:\mathds{N}. i_n \leq x$. A proposição pode ser reescrita da forma:
\[P(k) \triangleq i_k \leq x\]

\paragraph{Caso base} Precisamos provar que $P(0) \triangleq i_0 \leq x$
Veja que:
\begin{proofbox}
  \:x > 0     \= PRE \\
  \:< 1       \= Arit.   \\
  \:\geq i_0  \= i:=1   \\
\end{proofbox}
\hfill $\square$

\paragraph{Caso indutivo} Assumimos $P(k) \triangleq i_k \leq x$ como Hipótese de Indução ($HI$). Logo, precisamos provar que: \[P(k+1) \triangleq i_{k+1} \leq x\]
Agora, veja que:
\begin{proofbox}
  \:i_k \leq x \land i_k \neq x \= HI\ e\ laço \\
  sse\:i_k < x                  \= Arit. \\
  sse\:i_k + 1 \leq x           \= Arit. \\
  sse\:i_{k+1} \leq x           \= i:=i+1 \\
\end{proofbox}
\hfill $\square$

\subsubsection{Prova da Tripla de Hoare}

Desejamos, agora, provar a correção parcial da Tripla de Hoare definida anteriormente. Para facilitar, algumas definições auxiliares:
\begin{align}
S1  &\triangleq i:=1;\ j:=4; \nonumber \\
S2  &\triangleq while\ i \neq x\ \{ S \} \nonumber \\
S   &\triangleq Sa;\ Sb \nonumber \\
Sa  &\triangleq j:=j+2*i+3 \nonumber \\
Sb  &\triangleq i:=i+1 \nonumber \\
PRE &\triangleq x > 0 \nonumber \\
POS &\triangleq j = (x + 1)^2 \land \lnot i \neq x \nonumber \\
INV &\triangleq j = (i + 1)^2 \land i \leq x \nonumber
\end{align}

Em um primeiro momento desejamos provar a correção parcial de $(|\ PRE\ |)\ S1\ (|\ INV\ |)$. Agora, veja que:
\begin{proofbox}
  \:(|\ i \leq x\ |)\ i:=1\ (|\ 4 = (i + 1)^2 \land i \leq x\ |)\= Atrib. \\
  \:(|\ 4 = (i + 1)^2 \land i \leq x\ |)\ j:=4\ (|\ j = (i + 1)^2 \land i \leq x\ |) \= Atrib.  \\
  \:(|\ i \leq x\ |)\ S1\ (|\ INV\ |) \= Comp(1,2)  \\
  \(\:x > 0 \= Hip., PRE \\
    \:x \geq 1 \= Arit. \\
    \:x \geq i \= i:=1 \\
    \:i \leq x \= Simetria\ do\ operador 
  \)\\
  \:PRE \rightarrow i \leq x \= \intro\rightarrow(4-7) \\
  \:(|\ PRE\ |)\ S1\ (|\ INV\ |) \= PreStren(8,3) \\
\end{proofbox}
\hfill $\square$

Agora, desejamos provar que $(|\ INV\ |)\ S2\ (|\ POS\ |)$. Então, veja:
\begin{proofbox}
  \:(|\ j + 2i + 3 = (i + 2)^2 \land i + 1 \leq x\ |)\ Sa\ (|\ j = (i + 2)^2 \land i + 1 \leq x\ |)\= Atrib. \\
  \:(|\ j = (i + 2)^2 \land i + 1 \leq x\ |)\ Sb\ (|\ INV\ |) \= Atrib.  \\
  \:(|\ j + 2i + 3 = (i + 2)^2 \land i + 1 \leq x\ |)\ S\ (|\ INV\ |) \= Comp(1,2)  \\
  \(\: j = (i + 1)^2 \land i \leq x \land i \neq x \= Hip., PRE \\
    \: j = (i + 1)^2 \land i < x \= Def.\ \leq \\
    \: j + 2i + 3 = (i + 1)^2 \land i < x \= j := j + 2 * i + 3 \\
    \: j + 2i + 3 = (i + 2)^2 \land i + 1 < x \= i := i + 1 \\
    \: j + 2i + 3 = (i + 2)^2 \land i + 1 \leq x \=\ PRE\ (x > 0)
  \)\\
  \:INV \land i \neq x \rightarrow j + 2 * i + 3 = (i + 2)^2 \land i + 1 \neq x \= \intro\rightarrow(4-8) \\
  \:(|\ INV \land i \neq x \ |)\ S\ (|\ INV\ |) \= PreStren(9,3) \\
  \:(|\ INV\ |)\ S2\ (|\ POS\ |) \= While(10) \\
\end{proofbox}
\hfill $\square$

Por fim, podemos fazer a composição com o provado nas duas provas anteriores:
\begin{proofbox}
  \:(|\ PRE\ |)\ S1 \ (|\ INV\ |)\= Prova\ \#1 \\
  \:(|\ INV\ |)\ S2 \ (|\ POS\ |)\= Prova\ \#2 \\
  \:(|\ PRE\ |)\ S1; S2\ (|\ POS\ |) \= Comp(1,2)  \\
\end{proofbox}
\hfill $\square$

%%%%%%%%%%%%%%%%%%%%%%%%%%%%%%%%%%%%%%%%%%%%%%%%%%%%%%%%%%%%%%%%%%%%%%%%%
%%%%%%%%%%%%%%%%%%%%%%% PARTE 2 %%%%%%%%%%%%%%%%%%%%%%%%%%%%%%%%%%%%%%%%%
%%%%%%%%%%%%%%%%%%%%%%%%%%%%%%%%%%%%%%%%%%%%%%%%%%%%%%%%%%%%%%%%%%%%%%%%%

\section{Algoritmo 02}\label{sec:algo1}
\begin{lstlisting}
$Algoritmo\ 02\ (inteiros\ x,\ y,\ natural\ n)$
$begin$
$variaveis\ locais$
$\ \ \ \ inteiros\ i,\ j;$
$\ \ \ \ i := 0; j := x;$
$\ \ \ \ while\ i \neq n$
$\ \ \ \ \ \ \ \ j := j * y;$
$\ \ \ \ \ \ \ \ i := i + 1$
$end$
\end{lstlisting}

Através da definição do cabeçalho do programa, podemos inferir a pré-condição da Tripla de Hoare correspondente: $(|\ n \geq 0\ |)$. Ainda, através de um teste de
mesa podemos levantar hipóteses a respeito do que o programa faz e a invariante do laço.

\begin{table}[h]
\begin{tabular}{|l|l|l|l|l|l|l|l|}
\hline
$k$ & $i$ & $j$   & $x$ & $y$ & $n$ & $i \neq n$ & $j=xy^i \land i \leq n $ \\ \hline
$0$ & $0$ & $2$   & $2$ & $3$ & $5$ & $0 \neq 5$ & $2=2   \times 3^0 \land 0 \leq 5 $ \\ \hline
$1$ & $1$ & $6$   & $2$ & $3$ & $5$ & $1 \neq 5$ & $6=2   \times 3^1 \land 1 \leq 5 $ \\ \hline
$2$ & $2$ & $18$  & $2$ & $3$ & $5$ & $2 \neq 5$ & $18=2  \times 3^2 \land 2 \leq 5 $ \\ \hline
$3$ & $3$ & $54$  & $2$ & $3$ & $5$ & $3 \neq 5$ & $54=2  \times 3^3 \land 3 \leq 5 $ \\ \hline
$4$ & $4$ & $162$ & $2$ & $3$ & $5$ & $4 \neq 5$ & $162=2 \times 3^4 \land 4 \leq 5 $ \\ \hline
$5$ & $5$ & $486$ & $2$ & $3$ & $5$ & $5  =   5$  & $486=2 \times 3^5 \land 5 \leq 5 $ \\ \hline
$-$ & $5$ & $486$ & $2$ & $3$ & $5$ & $0 \neq 5$ & $486=2 \times 3^5 \land 5 \leq 5 $ \\ \hline
\end{tabular}
\end{table}

Analisando a tabela gerada, fica evidente que o programa tem, como objetivo, realizar o cálculo de $x*y^n$ e armazenar o resultado na variável $j$. Disto podemos concluir que a pós-condição
da Tripla de Hoare é $(|\ j = x*y^n\ |)$. Além disto, encontramos um bom candidato para a invariante do laço: $j = x * y^i \land i \leq n$. Para ter certeza que a
invariante é válida, é necessário realizar a prova formal da mesma.

\subsection{Tripla de Hoare}

\begin{lstlisting}
$(| n \geq 0 |)$
$i := 0; j := x;$
$while\ i \neq n$
$\ \ \ \ j := j * y;$
$\ \ \ \ i := i + 1$
$(| j = x*y^n |)$
\end{lstlisting}

\subsection{Prova da Invariante}\label{sec:algo1:invar}
Devemos provar que a invariante permanece válida durante a execução do laço, não importando o número de iterações. Ou seja, devemos provar que:
\[\forall n:\mathds{N}. j_n = x * y^i_n \land i_n \leq n\]

Aplicando o teorema da distributividade do operador $\forall$\footnote{$\forall x.P(x) \land Q(x) = \forall x.P(x) \land \forall x.Q(x)$},
podemos dividir a prova da invariante em duas partes: $\forall n:\mathds{N}. j_n = x * y^i_n$ e $\forall n:\mathds{N}. i_n \leq n$.

À partir disto, podemos aplicar técnicas de indução natural.

\subsubsection{Prova - Parte I}

\subsubsection{Prova - Parte II}

\subsubsection{Prova da Tripla de Hoare}

\section{Conclusão}
\paragraph{}
Insira conclusão aqui.
É eeeeeasy. Tendeuquequedizê? Mecânicow.

\bibliographystyle{sbc}
\bibliography{teste}

\end{document}
